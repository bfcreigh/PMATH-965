\documentclass[main.tex]{subfiles}

\begin{document}

\lbl{Recall}
\begin{itemize}
    \item  A \emphp{connection} on a vector bundle $(E, B, \pi, \R^r)$ is a map $D : \sect{E} \to \sect{T^*B \otimes E}$ that is $\R$-linear and satisfies $D(f\sigma) = df \otimes \sigma + fD(\sigma)$ for any $f \in C^\infty(B)$ and $\sigma \in \sect{E}$.

    \item Given an atlas $\brc{ \br{U_\alpha, \varphi_\alpha} }$ of $E$ and local frames $e_i^\alpha = \varphi_{\alpha}^{-1}(-, \vec{e}_i)$,
    \[
    D(e_j^\alpha) = \sum_i \alpha_{ij}^\alpha \ot e_i
    \]
    where $a_{ij}^\alpha \in \Omega^1(U_\alpha)$, so that $A_\alpha = (a_{ij}^\alpha)$ is a matrix of $1$-forms, called the connection matrix of $D$ over $U_\alpha$.

    \lbl{Claim} For all $b \in U_\alpha \cap U_\beta \neq \emptyset$,
\[
e_j^\beta(b) = \sum_i \br{\ol{g}_{\alpha\beta}(b)}_{ij}e_i^\alpha(b).
\]

\begin{proof}
\al{e_j^\beta(b) &= \varphi_\beta^{-1}(b, \vec{e}_j)\\
                 &= \varphi_\alpha^{-1} \circ g_\alpha\beta(b, \vec{e}_j) \\
                 &= \varphi_\alpha^{-1}(b, \ol{g}_{\alpha\beta}(b)\vec{e}_j) \\
                 &= \sum_{i} \br{ \ol{g}_{\alpha\beta}(b) }_{ij} \varphi_\alpha^{-1}(b, \vec{e}_i) \\
                 &= \sum_{i} \br{ \ol{g}_{\alpha\beta}(b) }_{ij} e_i^\alpha.
                 }
\end{proof} So the change of basis matrix from $\brc{e_1^\alpha, \dots, e_r^\alpha}$ to $\brc{ e_1^\beta, \dots, e_r^\beta }$ is $\ol{g}_{\alpha\beta}$, so
\[
A_\beta = \ol{g}_{\alpha\beta}^{-1} d \ol{g}_{\alpha\beta} + \ol{g}_{\alpha\beta} A_\alpha \ol{g}_{\alpha\beta}.
\]

\item $\mc{A}(E) = \brc{ \text{all connections on $E$ }}$ is not closed under addition. Nonetheless, it is convex: For all $D_1, \dots, D_l \in \mc{A}(E)$ and $a_1, \dots, a_l \in \R$ such that $\sum_{j=1}^l a_j = 1$, we have that
\[
    a_1D_1 + \dots + a_lD_l \in \mc{A}(E).
\]
\end{itemize}

\begin{prop}
\mc{A}(E) is an affine space modeled on $\Omega^1\br{\text{End}(E)} := \sect{T^*M \ot \text{End}(E)}$.
\end{prop}

\begin{note}
Let $(V, B, \pi, \R^r)$ be a vector bundle and set $\Omega^k(V) := \sect{\bigwedge^k B \ot V}$. Locally, $\tau \in \Omega^k(V)$ looks like $\tau = \sum_{i=1}^m \omega_i \ot e_i$ where $\brc{e_1,\dots, e_m}$ is a local frame of $V$ and $\omega_1, \dots, \omega_m \in \bigwedge^{k}U$ with $U \subseteq B$ open. For any $X_1, \dots, X_k \in \sect{TB}$, we define
\al{
\tau(X_1, \dots, X_k) &:= \sum_{i=1}^m \omega_i(X_1, \dots, X_k) \ot e_i \\
                      &= \sum_{i=1}^m \omega_i(X_1, \dots, X_k) e_i \in \sect{V}.
}
Note that the definition of $\tau(X_1,\dots, X_k)$ is independent of the local description of $\tau$.
\end{note}

\begin{proof}

    Let $D_0 \in \mc{A}(E)$. It is enough to show that
\[
\mc{A}(E) = \brc{ D_0 + a \s a \in \Omega^1\br{\text{End}(E)}  }
\]
What do elements of $\Omega^1(\text{End}(E))$ look like? Locally, $a = \sum_{i}a_i \ot \psi_i$ where the $a_i$ are $1$-forms and $\psi_i \in \text{End}(E\rst{U})$ where $U \subset B$ is open. Then for all $\sigma \in \sect{E\rst{U}}$,
\[
a(\sigma) = \sum_i a_i \ot \psi_i(\sigma)
\]
so
\al{
a: \sect{E} &\to \sect{T^*B \ot E} \\
  \sigma &\mapsto a(\sigma).
}
So $a$ is $C^\infty(B)$-linear because, for any $f \in C^\infty(B)$,
\al{
a(f\sigma) &= \sum_i a_ \ot \psi_i(f\sigma) \\
           &= \sum_i a_i \ot f\psi_i(\sigma) \\
           &= f \sum_{i} a_i \ot \psi_i(\sigma) \\
           &= f a(\sigma).
}
So any $a \in \Omega^1(\text{End}(E))$ induces a $C^\infty(B)$-linear map $a : \sect{E} \to \sect{T^*B \ot E}$. Conversely, any $C^\infty(B)$-linear map $a: \sect{E} \to \sect{T^*B\ot E}$ induces an element of $\Omega^1(\text{End}(E))$.

Let $D, D^\prime \in \mc{A}(E)$. Let us check that \[
D - D^\prime \in \Omega^1(\text{End}(E)).
\]
It is enough to check that the induced map
\al{D - D^\prime : \sect{E} &\to \sect{T^*B \ot E} \\
\sigma &\mapsto D(\sigma) - D^\prime(\sigma)
}is $C^\infty(B)$-linear. let $\sigma, \sigma^\prime \in \sect{E}$ and $f \in C^\infty(B)$. Then
\al{
(D - D^\prime)(f\sigma + \sigma^\prime) &= \br{ D(f\sigma) + D(\sigma^\prime) } - \br{ D^\prime(f\sigma) + D^\prime(\sigma^\prime) } \\
&= \br{df \otimes \sigma + f D(\sigma) + D(\sigma^\prime) } - \br{ df \otimes \sigma + f D^\prime(\sigma) - D^\prime(\sigma^\prime) } \\
&= f(D - D^\prime)(\sigma) + (D - D^\prime)(\sigma^\prime).
}
and so $D - D^\prime \in \Omega^1(\text{End}(E))$.

\end{proof}

We have seen that connections generalize the exterior derivative.

\lbl{Recall} Let $U \subset B$ be open with coordinates $(x_1, \dots, x_n)$. Then for any $f \in C^\infty(U)$, then
\[
df = \sum_{i=1}^n \frac{\partial f}{\partial x_i} dx_i.
\]
In particular, if for any $i \in \{1, \dots, n\}$, we geta
\[
df \br{\frac{\partial}{\partial x_i}} = \frac{\partial f}{\partial x_i}.
\]
In general, for any $X = \sum_{i} a_i \frac{\partial}{\partial x_i}$, then
\[
df(X) = \sum_i a_i \frac{\partial f}{\partial x_i} = \nabla f \cdot (a_1, \dots, a_n).
\]
Also, for all $\omega \in \Omega^1(U)$,
\[
\omega = \sum_i \omega\br{\frac{\partial}{\partial x_i}} dx_i.
\]

Lets go back to a connection $D \in \mc{A}(E)$. Let $U \subset B$ be an open set over which $B$ has coordinates $x_1, \dots, x_n$ and $E$ is trivial with local frame $\brc{e_1, \dots, e_r}$. Then for all $\sigma \in \sect{E\rst{U}}$,
\[
D(\sigma) = \sum_{i=1}^r \omega_i \ot e_i
\]
with $\omega_i \in \Omega^1(U)$. And, for all $X \in \sect{TU}$,
\[
D(\sigma)(X) := \sum_{i=1}^r \omega_i(X) e_i \in \sect{E\rst{U}}.
\] So for fixed $X \in \sect{TB}$, we get a map
\al{
D_X : \sect{E} &\to \sect{E} \\
        \sigma &\mapsto D(\sigma)(X)
}
Note that $D_X$ is $\R$-linear and satisfies Leibniz in $\sigma$. We say that $D_X(\sigma)$ is the \emphp{covariant derivative of $\sigma$ in the direction $X$}. Also note that for any $f \in C^\infty(B)$,
\[
D(\sigma)(fX) = f\br{ D(\sigma)(X) }, \text{ or } D_{fX}(\sigma) = f D_X(\sigma).
\]
We then get a map
\al{
\nabla : \sect{TB} \times \sect{E} &\to \sect{E} \\
            (X, \sigma) &\mapsto D_X(\sigma)
}
such that it is
\begin{itemize}
    \item $C^\infty(B)$-linear in $X$
    \item $\R$ linear in $\sigma$
    \item Satisfies Leibniz in $\sigma$:
    \al{
    D_X(f\sigma) &= D(f\sigma)(X) \\
                 &= \br{df \ot \sigma + fD(\sigma)}(X) \\
                 &= df(X)\sigma + fD(\sigma)(X) \\
                 &= X(f)\sigma + fD_X(\sigma).
    }
\end{itemize}

\begin{defn}
A map $\nabla: \sect{TB}\times\sect{E} \to \sect{E} $ such that
\begin{itemize}
    \item $C^\infty(B)$-linear in $X$,
    \item $\R$-linear in $\sigma$, and
    \item $\nabla(X, f\sigma) = X(f)\sigma + f\nabla(X , \sigma)$
\end{itemize} is called a \emphp{linear connection on $E$}, or a \emphp{covariant derivative on $E$}.
\end{defn}


\begin{note}
    \begin{enumerate}
        \item Tu defines connections this way.
        \item There is a one-to-one correspondence between elements of $\mc{A}(E)$ and linear connections  $\nabla: \sect{TM} \times \sect{E} \to \sect{E}$. We saw that any $D \in \mc{A}(E)$ induces a $\nabla$. Conversely, given a linear connection $\nabla$, we can define $D \in \mc{A}(E)$ by
        \al{
        D: \sect{E} &\to \sect{T^*B \ot E} \\
            \sigma &\mapsto \nabla(-, \sigma)
        }
        \item When $E = TB$, linear connections
        \al{
        \nabla: \sect{TB} \times \sect{TB} \to \sect{TB}
        }
        are called \emphp{affine connections}. In local coordinates $(x_1, \dots, x_n)$ on $B$ and a local frame $\brc{e_1, \dots, e_r}$ on $E$:
        \al{
        D(\sigma) &= \sum_i \omega_i \ot e_i \acom{with $\omega_i \in \bigwedge^1 (U)$} \\
                  &= \sum_{i,j} \omega_i\br{\frac{\partial}{\partial x_j} dx_j \ot e_i} \\
                  &= \sum_j dx_j \ot \br{ \sum_i \omega_i\br{\frac{\partial}{\partial x_j} e_j}} \\
                  &= \sum_j dx_j \ot D_{\frac{\partial}{\partial x_j}}(\sigma)
        }
    \end{enumerate}
\end{note}

\end{document}
