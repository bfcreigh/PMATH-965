\documentclass[main.tex]{subfiles}

\begin{document}
\lbl{Sections} $(E, B, \pi, F)$ a fibre bundle. A section is a smooth map $\sigma: B \to E$ such that $\pi \circ \sigma = \id{B}$. We denote by $\sect{E}$ the set of all sections of $(E, B, \pi, F)$.

Gien a bundle chart $\br{ E_U, \varphi_U}$ with $U \subseteq B$ open,
\[
\begin{tikzcd}
\varphi_U \circ (\sigma \big|_U) : U \arrow[rd, "\sigma"'] \arrow[rr] &                              & U \times F \\
                                                                      & E_U \arrow[ru, "\varphi_U"'] &
\end{tikzcd}
\]
with $\varphi_U \circ (\sigma\rst{U})(b) = (b, \overline{\sigma}(b))$ for some smooth $\overline{\sigma}: U \to F.$

Let $\brc{U_\alpha}_{\alpha \in \mc{A}}$ be an open conver of $B$
 and $\brc{ \br{E_{U_\alpha}, \varphi_\alpha} }_{\alpha \in \mc{A}}$ be
 a bundle atlas for $(E, B, \pi, F)$. Let $\sigma \in \sect{E}$. Set
 \[
 \sigma_\alpha := \sigma\rst{U_\alpha}: U_\alpha \longrightarrow E_{U_\alpha} = \coprod_{b \in U_\alpha} E_b
 \]
 Then
 \al{\varphi_\alpha \circ \sigma_\apha : U_\alpha &\to U_\alpha \times F \\
 b &\mapsto (b, \overline{\sigma}_\alpha(b))}

 for some smooth $\overline{\sigma}_\alpha: U_\alpha \to F$. How are the $\overline{\sigma}_\alpha$'s related? Suppose $U_\alpha \cap U_\beta \neq \emptyset$  and let $b \in U_\alpha \cap U_\beta$. Then
 \al{
 \br{ b, \overline{\sigma}_\alpha(b) } &= \varphi_\alpha \circ \sigma_\alpha(b) \\
 &= \varphi_\alpha \circ \sigma_\beta(b) \\
 &= \underbrace{\varphi_\alpha \circ \varphi_\beta^{-1}}_{g_{\alpha\beta}} \circ \varphi_\beta \circ \sigma_\beta(b) \\
 &= \br{ b, \overline{g}_{\alpha\beta}(b)\br{ \overline{\sigma}_\beta(b) }}
 }
 which implies that
 \[
    \overline{\sigma}_\beta(b) = \overline{g}_{\alpha\beta}(b)\br{\overline{\sigma}_\beta(b) } (***)
 \] for all $b \in U_\alpha \cap U_\beta$.

 So, given a bundle atlas $\brc{ \br{ E_{U_\alpha}, \varphi_\alpha }  }$ of $(E, B, \pi, \alpha)$, we can think of sections of the bundle as families of smooth maps $\brc{ \sigma_\alpha : U_\alpha \to F}$ that satisfy $(***)$.

 \subsection{Sections of Vector Bundles}

 Let $(E, B, \pi, \R^r)$ be a vector bundle, which we will denote by $E$. Let $\brc{U_\alpha}$ be an open cover of $B$ and $\brc{ \br{ E_{U_\alpha}, \varphi_\alpha  }}$ be a vector bundle atlas of $E$. Then, the transition functions of the atlas are
 \al{
 \overline{g}_{\alpha\beta}: U_\alpha \cap U_\beta &\to \gl{r}{\R}
 }
 So, for all $b \in U_\alpha \cap U_\beta$, $\overline{g}_{\alpha\beta}(b) = \br{\text{invertible matrix} }$, and, for all $v \in \R^r$,
 \[
 \overline{g}_{\alpha\beta}(b)(v) = \underbrace{\overline{g}_{\alpha\beta}(b) \cdot v}_{\text{matrix multiplication}}
 \]
 For this reason, $\overline{g}_{\alpha\beta}(b)$ are sometimes called \emphp{transition matrices}.

 Also, any section of $E$ is determined by a family \[
 \brc{\overline{\sigma}_\alpha : U_\alpha \to \R^r}
 \]
 of smooth vector-valued functions such that
 \[
 \overline{\sigma}_\alpha(b) = \underbrace{\overline{g}_{\alpha\beta}(b) \cdot \sigma_\beta(b)}_{\text{matrix multiplication}}
 \] by $(***)$.

 \begin{note}
 On a vector bundle, any local section can be extended globally (possibly by zero outside of the open set on which it is defined) by using bump functions (exercise).
 \end{note}

 \begin{defn}
 Let $\sigma_1, \dots, \sigma_l \in \sect{E}$. We say that the set $\brc{\sigma_1, \dots, \sigma_l}$ is \emphp{linearly independent} if
 \[
 \brc{\sigma_1(b), \dots, \sigma_l(b)} \subseteq E_b
 \] is linearly independent for every $b \in B$. If $l = r$ (the rank of $E$), then $\brc{\sigma_1, \dots, \sigma_l}$ is called a \emphp{frame for $E$}.
 \end{defn}

 \begin{note}
 \begin{enumerate}[(i)]
     \item If $\brc{\sigma_1, \dots, \sigma_r }$ is a frame of $E$ so that $\brc{\sigma_1(b),\dots, \sigma_l(b)}$ is linearly independnet in $E_b$ for all $b \in B$, then $\brc{ \sigma_1(b), \dots, \sigma_l(b) }$ is a basis for $E_b$ for all $b \in B$. Then $\sigma_i(b)  \neq 0$ for all $i = 1, \dots, l$. So, the $\sigma_i$'s are nowhere-vanishing.
     \item If $r=1$, then any frame of $E$ consists solely of a nowhere-vanishing section.
 \end{enumerate}
 \end{note}

 \begin{exmp}
 \begin{enumerate}[1)]
     \item Let $S^{2n}$ be an even-dimensional sphere. Then, by the Hairy Ball theorem, any tangent vector field of $S^{2n}$ has at least one zero. Thus, $TS^{2n}$ does not admit nowhere-vanishing sections. So, $TS^{2n}$ does not admit any (global) frames.
     \item $S^{2n+1} \subset \R^{2n+2} = \brc{ \br{x_1, \dots, x_{2n+2} }}$.
     \begin{itemize}
         \item $S^1 = \brc{(x_1, x_2) \in \R^2 \s x_1^2 + x_2^2 = 1}$. Then $X_{\br{x_1, x_2}} = (-x_2, x_1)$ is a nowhere-vanishing, tangent vector field of $S^1$.
         \item On $S^{2n+1} \subset \R^{2n+2}$, We define
         \[
         X_{\br{x_1, \dots, x_{2n+2}}} = \br{ -x_2, x_1, \dots, -x_{2k}, x_{2k+1}, \dots, -x_{2n-1}, x_{2n+2}}.
         \]
         \item On $S^3$, we have that
         \al{
         X_1(x_1, \dots x_4) &= (-x_2, x_1, -x_3, x_4) \\
         X_2(x_1, \dots x_4) &= (-x_3, -x_4, x_1, x_2) \\
         X_3(x_1, \dots x_4) &= (x_4, -x_3, -x_2, x_1) \\
         } comprise a frame for $TS^3$.
         \item On $S^7$, one can use the octonions to construct a frame for $TS^7$
         \item On $S^{2n+1}$ for $n \geq 4$, $TS^{2n+1}$ does not admit a global frame.
     \end{itemize}
     \item Let $E = B \times \R^r$ be the trivial vector bundle with $\pi(b, v) = b$. Then suppose that $\brc{e_1, \dots, e_r}$ is the standard basis for $\R^r$. Then a global frame is given by $\brc{\sigma_1, \dots, \sigma_r}$ where
     \al{
        \sigma_i : B &\to E \\
                   b &\mapsto (b, e_i).
     } We will refer to this as the \emphp{standard frame on the trivial bundle}. So, the trivial bundle admits at least one frame (in fact... many).
 \end{enumerate}
 \end{exmp}

 In general, we have:

 \begin{prop}
 A vector bundle $E$ is trivial if and only if it admits a frame.
 \end{prop}

 \begin{proof}
    \impliespf If $E$ is trivial, then it is isomorphic to $B \times \R^r$. Thus, there exists a vector bundle isomorphism $H: B \times \R^r \to E$. In particular, $H\rst{\brc{b}\times \R^r} \to E_b$ is a linear isomorphism. Let $\brc{\sigma_1, \dots, \sigma_r}$ be the standard frame on $B \times \R^r$, and define
    \al{
    \tilde{\sigma}_i : B &\to E \\
                       b &\mapsto H \circ \sigma_i(b).
    }
    Then each $\tilde{\sigma}_i$ is a section of $E$, because $\pi \circ \tilde{\sigma}_i = \pi \circ H \circ \sigma_i = \text{proj}_1 \circ \sigma_i = \id{B}$. Also, for all $b \in B$, \[\brc{\tilde{\sigma}_1(b), \dots, \tilde{\sigma}_r(b)} = \underbrace{H\rst{b} \br{\brc{\sigma_1(b), \dots, \sigma_r(b)}}}_{\text{linearly independent}}.\] So $\brc{\tilde{\sigma}_1, \dots, \tilde{\sigma}_r(b)}$ is a frame for $E$. \impliedpf Assume that $E$ admits the frame $\brc{\sigma_1, \dots, \sigma_r}$ and use it to construct an isomorphism given by
    \al{
        H: B \times \R^r &\to E \\
        (b, (a_1, \dots, a_r)) &\mapsto \sum_{i=1}^r a_i \sigma_i(b) \in E_b,
    } which is an isomorphism because $\brc{\sigma_1, \dots, \sigma_r}$ is a frame. So, $H$ is a vector bundle isomorphism.
 \end{proof}

 \begin{cor}
 A line bundle is trivial if and only if it admits a nowhere-vanishing section.
 \end{cor}

 \begin{cor}
 $TS^k$ is trivial if and only if $k \in \brc{1,3,7}$.
 \end{cor}

\begin{defn}
 A manifold $M$ is called \emphp{parallelizable} if its tangent bundle is trivial.
\end{defn}

\begin{exmp}
    \begin{enumerate}
        \item $S^1, S^3, S^7$ are parallelizable.
        \item Any Lie group $G$ is parallelizable.
    \end{enumerate}
\end{exmp}

\begin{prop}
 The tautological line bundle on $\mb{P}^n$ is not trivial.
\end{prop}
\begin{proof}
The tautological line bundle on $\mb{P}^n$ does not admit any nowhere-vanishing sections.
\end{proof}


 \end{document}
